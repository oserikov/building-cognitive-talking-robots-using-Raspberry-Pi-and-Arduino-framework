\prettypart{ЗАКЛЮЧЕНИЕ}
% \section{Положения, выносимые на защиту}
% \section{Результаты работы}

В работе был рассмотрен и исследован реактивно"=делиберативный подход к созданию когнитивных роботов. Практическим результатом работы является разработанный программно"=аппаратный комплекс для создания диалоговых когнитивных роботов на основе реактивно"=делиберативного подхода. 

Разработанный программно"=аппаратный комплекс был использован в когнитивных диалоговых роботах <<Подушка>> компании МеханиУм. Роботы были представлены на мероприятиях 

\begin{itemize}
    \item Российский финал Microsoft Imagine Cup 2018,
    \item Skolkovo Robotics Forum 2018
    \item RoboCity 2018 в г.\,Одинцово
    \item Фестиваль <<Политех>>
\end{itemize}

Один из роботов <<Подушка>> стал экспонатом музея <<Открытый космос>>. 


Сформулированная в выпускной квалификационной работе задача была полностью решена в рамках выпускной квалификационной работы, разработанное решение, однако, подлежит дальнейшей доработке, разностороннему развитию. 

% \section{Выводы, сделанные в процессе работы}
% \section{Предположения, сделанные в процессе работы}

Дальнейшее развитие разработанного в ходе ВКР решения состоит в первую очередь в завершении начатых масштабных процессов публикации решения, как готового продукта с открытым исходным кодом. На этом этапе предстоит решить задачи лицензирования решения и написания документации. 
Также полежит завершению начатый рефакторинг кода, призванный облегчить дальнейшую поддержку и сопровождение решения.

% \section[Перспективы применения результатов работы \\на практике]{Перспективы применения результатов работы на практике}
% \section{Возможности дальнейшего исследования проблемы}