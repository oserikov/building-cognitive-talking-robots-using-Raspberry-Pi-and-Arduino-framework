\chapter{Теоретическая часть работы}

\section{Исследование решений, существующих в области объекта работы}

Объектом работы явился когнитивный диалоговый робот как единица образовательной робототехники. Выбор объекта был произведён вследствие исследования существующих единиц образовательной робототехники.

\subsection{Существующие решения в образовательной робототехнике}

Исследование СМИ, сети интернет и центров образовательной робототехники, расположенных в г.\,Москве показало, что сейчас образовательная робототехника состоит в подавляющем большинстве из роботов --- тренажёров для программистов. Представляя наглядную демонстрацию результатов работы программ, эти роботы не несут образовательной нагрузки в смысле осуществления процесса передачи информации --- они выступают в роли инструмента проверки уже усвоенных знаний. 

Популярны также конструкторы, дополненные программируемыми электронными компонентами и, наоборот, электронные компоненты, снабжённые наборами добавляемых внешних механизмов. Такие наборы часто сопровождаются вспомогательной литературой в качестве основного инструмента донесения образовательной нагрузки до конечного обучающегося.

Образовательный процесс как диалог учителя с учеником автоматизирован только в виртуальной среде --- недавно стали появляться чат"=боты и образовательные программы. Популярных, доступных роботизированных решений в реальном мире в сфере диалогового образования не существует.

\section{Инструменты, необходимые для роботизации диалога}\label{dialogue-instruments}
В диалоге можно выделить две фундаментальные части, автоматизация которых составит значительную часть автоматизации самого диалога. Во"=первых, так как диалог всегда ведётся на некотором языке, необходим этот самый язык. Во"=вторых, диалог можно рассматривать как последовательность взаимных реакций.

\subsection{Язык в автоматизированном диалоге}\label{language-auto}
Из определения, приведённого в энциклопедии <<Кругосвет>> \cite{lang-krugosvet} следует, что, среди множества факторов определяющих язык, одним из ключевых факторов является цель общения. 

Так как разрабатываемое решение, реализующее помимо прочего инструмент диалога, призвано быть доступным и гибким, конечная цель общения в рамках разработанного решения заранее неизвестна и должна быть определена конечным пользователем решения. Отсюда вытекает необходимость предоставить средства для определения конечного языка конечному пользователю. 





\subsection{Автоматизация реакций}
Диалог можно рассматривать как последовательность реакций, например начало диалога --- это реакция на создание условий, способствующих началу диалога. В процессе диалога ответные реплики являются реакциями на некоторые события, приводящие диалог в некоторое состояние. В процессе диалога можно наблюдать как осмысленные реакции, продиктованные целью общения, так и рефлекторные почти мгновенные реакции на какие"=то события.



\section{Элементы аппаратной платформы роботизированного решения} \label{apparat-platform}

Две выделенные в \ref{dialogue-instruments} задачи многократно реализованы программно, поэтому любая аппаратная платформа, обладающая возможностью исполнения программ, написанных на языках общего назначения, сможет так или иначе реализовать роботизацию диалога.  

Как сказано в \ref{popular-instruments}, популярными инструментами доступной робототехники являются микроконтроллеры Arduino и микрокомпьютеры Raspberry Pi. Желание сделать простое в использовании, настройке и сопровождении решение мотивирует ориентировать решение на работу с этими инструментами. 


Подраздел \ref{language-auto} поднял вопрос реализации средств определения языка общения участников диалога. Реализуя связь микроконтроллеров
Arduino и микрокомпьютеров Raspberry Pi, предстоит также реализовать язык взаимодействия этих двух классов устройств, который будет основываться на том или ином средстве передачи данных. Последняя задача требует исследования средств ввода"=вывода, доступных в микроконтроллерах Arduino и микрокомпьютерах Raspberry Pi.

\subsubsection{Средства ввода"=вывода в микроконтроллере Arduino}
Различные продукты семейства Arduino имеют различный набор средств ввода-вывода, гарантированно же присутствуют только ПИНЫ и USB. Популярным дополнительным средством является ethernet.

Плюсы ethernet

Минусы ethernet

Плюсы USB

Минусы USB

Плюсы Пинов

Минусы пинов

\subsubsection{Средства ввода"=вывода в микрокомпьютере Raspberry Pi}
Все продукты семейства Raspberry имеют ПИНЫ, USB, ethernet.

Плюсы ethernet

Минусы ethernet

Плюсы USB

Минусы USB

Плюсы Пинов

Минусы пинов

\section{Подходы к автоматизации диалогов}
Автоматизацию диалога можно рассмотреть как подзадачу искусственного интеллекта. Множество различных идей о решении задач искусственного интеллекта можно разделить на два подхода: \textbf{\textit{подход первый и подход второй}}.

Последний подход основан на моделировании функционирования реальных биологических элементов --- особенно популярным в наши дни примером такого подхода являются нейронные сети. Для решения задачи, поставленной в выпускной квалификационной работе, такой подход кажется мало подходящим потому, что поставленная задача не состоит в моделировании поведения биологических элементов вовсе.

Первый же подход к моделированию интеллектуальных систем, состоящий в нисходящем моделировании высокоуровневых интеллектуальных процессов, кажется подходящим для решения поставленной задачи, состоящей как раз в моделировании высокоуровневого интеллектуального образовательного процесса. Рассмотрим этот подход подробнее.

\subsection{Нисходящий подход в разработке решений искусственного интеллекта}

{\Large Про когнитивность}

