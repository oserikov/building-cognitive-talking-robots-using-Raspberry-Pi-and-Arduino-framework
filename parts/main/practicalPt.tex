\chapter{Практическая часть работы}

Вышеописанные исследования привели к чётко поставленной задаче --- реализовать программно-аппаратный комплекс, который позволял бы максимально доступным образом создавать диалоговых когнитивных роботов. Программная часть должна реализовывать логику робота, используя реактивно"=делиберативный подход. Аппаратная часть комплекса должна быть основана на микроконтроллерах Arduino и микрокомпьютерах Raspberry Pi. В \ref{apparat-platform} была поставлена задача реализовать взаимодействие двух устройств: Arduino и Raspberry Pi. Было принято решение основать взаимодействие устройств на обмене командами из множества определяемых пользователем команд.

\section{Основные этапы в разработке решения задачи, сформулированной в выпускной квалификационной работе}
Основными этапами разработки явились:
\begin{enumerate}
\item подготовка аппаратной платформы решения 
\begin{itemize}
    \item исследование микроконтроллеров Arduino и микрокомпьютеров Raspberry Pi, представляющих основу аппаратной части, с точки зрения программиста
    \item разработка единой аппаратной платформы, объединяющей Arduino и Raspberry Pi в единый элемент, реализующий поставленную задачу
\end{itemize}

\item разработка необходимого программного обеспечения
\begin{itemize}
    \item выбор технологий для построения программного решения поставленной задачи
    \item выделение ключевых подзадач поставленной задачи в самостоятельные единицы, подлежащие разработке
    \item разработка выделенных на предыдущем этапе элементов
\end{itemize}

\item отладка разработанного решения
\begin{itemize}
    \item отладка разработанного решения
\end{itemize}

\item доработка решения с учётом требований потенциальных пользователей
\begin{itemize}
    \item анализ разработанного решения потенциальными пользователями
    \item доработка разработанного решения с учётом требований потенциальных пользователей
\end{itemize}

\item применение разработанного решения к реальной задаче  
\begin{itemize}
    \item реализация пробного проекта вместе с потенциальными пользователями решения
    \item получение обратной связи от потенциальных пользователей разработанного решения и сторонних компетентных специалистов
\end{itemize}

\item публикация законченного разработанного решения как готового
\begin{itemize}
    \item внедрение разработанного решения в технологический стек потенциальных пользователей решения
    \item публикация исходного кода разработанного решения
\end{itemize}
\end{enumerate}

Этапы разработки подробно раскрыты ниже.

\section{подготовка аппаратной платформы решения}
\subsection{Выбор технологии, связывающей Arduino и Raspberry Pi}
В \ref{apparat-platform} были рассмотрены средства ввода"=вывода доступные как на устройствах Arduino, так и на устройствах Raspberry Pi.
Хотя оба класса устройств способны использовать три технологии ввода"=вывода (ПИНЫ, ethernet, USB), выбранная в качестве используемой технология (ПИНЫ!!1!) не явилась спорным решением.

Взаимодействие Arduino и Raspberry Pi на низком уровне запланировано быть реализованым не конечным пользователем решения, а разработчиками решения. Это мотивирует постараться выбрать максимально гибкое решение, которое было бы легко перенастраивать на самом низком уровне, если этого потребует процесс разработки --- разработчик, в отличие от конечного пользователя, готов исследовать сложную технологию для решения какой-то проблемы. 

Технологии ethernet и usb позволяют обмениваться большими объёмами данных и часто предоставляют программисту удобный высокоуровневый интерфейс для этого. Так как в поставленой задаче было принято решение организовать взаимодействие между Arduino и Raspberry Pi на основе некоторого множества определяемых команд, способность передавать большие объёмы данных между устройствами оказалась излишней. ПИНЫ11 же, предоставляющие простейший низкоуровневый способ обмена информацией побитно, прекрасно подошли для задачи.

Мощность множества команд взаимодействия Arduino и Raspberry Pi ограничивается сверху количеством ПИНЫ!!!.

Получившаяся в результате аппаратная платформа выглядит следующим образом: микроконтроллер Arduino и микрокомпьютер Raspberry Pi, общающиеся через ПИНЫ!!!.


\section{разработка необходимого программного обеспечения}
\subsection{Разделение задач между Arduino и Raspberry Pi}
Задачи, возникающие в процессе работы, были разделены между микроконтроллером Arduino и микрокомпьютером Raspberry Pi следующим образом: 
\begin{itemize}
    \item Rasbperry Pi \begin{itemize}
        \item реализует делиберативную составляющую диалога
        \item предоставляет интерфейс для реализации условий, проверяемых в процессе логического вывода
        \item принимает решения о выполнении тех или иных действий, требуемых текущим состоянием диалога
        \item предоставляет программный интерфейс для реализации подзадач общения с собеседником на стороне Raspberry Pi
    \end{itemize}
    
    \item Arduino \begin{itemize}
        \item реализует реактивную составляющую диалога
        \item предоставляет Raspberry Pi интерфейс для реализации подзадач общения с собеседником на стороне Arduino
    \end{itemize}
\end{itemize}

\subsection{Выбор ключевых элементов программного обеспечения}
\subsubsection{Ключевые элементы программного обеспечения Arduino}

\subsubsection{Выбор операционной системы для Raspberry Pi}
Семейство микрокомпьютеров Raspberry Pi может работать под управлением операционных систем, которые можно разбить на два класса: Unix"=подобные и Windows.

Исследование относительного объёма доступных информационных материалов по Arduino и Raspberry Pi показало, что несмотря на меньшую свободу действий, представляемую Windows, количество информации о разработке под Raspberry Pi, управляемую операционной системой Windows на порядок превосходит количество информации о разрабоке под Raspberry Pi, управляемую операционной системой *nix. 

\subsubsection{Архитектура разработанного программного обеспечения на стороне Raspberry Pi}
Программное обеспечение было реализовано в формате UWP"=приложения --- де"=факто стандартном формате для приложений под Windows IoT.

\section{отладка разработанного решения}
бла бла
\section{доработка решения с учётом требований потенциальных пользователей}
бла бла
\section{применение разработанного решения к реальной задаче}
бла бла
\section{публикация законченного разработанного решения как готового}
бла бла

\section{Обоснование принципов действия разработанных объектов}
\subsection{Архитектура разработанного решения}

