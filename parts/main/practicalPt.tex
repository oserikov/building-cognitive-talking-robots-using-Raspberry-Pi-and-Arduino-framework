\chapter{Практическая часть работы}

Вышеописанные исследования привели к чётко поставленной задаче --- реализовать программно-аппаратный комплекс, который позволял бы максимально доступным образом создавать диалоговых когнитивных роботов. Аппаратная часть комплекса должна быть основана на микроконтроллерах Arduino и микрокомпьютерах Raspberry Pi. Программная часть должна реализовывать логику робота, используя реактивно"=делиберативный подход.

\section{Основные этапы в разработке решения задачи, сформулированной в выпускной квалификационной работе}
Основными этапами разработки явились:
\begin{enumerate}
\item подготовка аппаратной платформы решения 
\begin{itemize}
    \item исследование микроконтроллеров Arduino и микрокомпьютеров Raspberry Pi, представляющих основу аппаратной части, с точки зрения программиста
    \item разработка единой аппаратной платформы, объединяющей Arduino и Raspberry Pi в единый элемент, реализующий поставленную задачу
\end{itemize}

\item разработка необходимого программного обеспечения
\begin{itemize}
    \item выбор технологий для построения программного решения поставленной задачи
    \item выделение ключевых подзадач поставленной задачи в самостоятельные единицы, подлежащие разработке
    \item разработка выделенных на предыдущем этапе элементов
\end{itemize}

\item отладка разработанного решения
\begin{itemize}
    \item отладка разработанного решения
\end{itemize}

\item доработка решения с учётом требований потенциальных пользователей
\begin{itemize}
    \item анализ разработанного решения потенциальными пользователями
    \item доработка разработанного решения с учётом требований потенциальных пользователей
\end{itemize}

\item применение разработанного решения к реальной задаче  
\begin{itemize}
    \item реализация пробного проекта вместе с потенциальными пользователями решения
    \item получение обратной связи от потенциальных пользователей разработанного решения и сторонних компетентных специалистов
\end{itemize}

\item публикация законченного разработанного решения как готового
\begin{itemize}
    \item внедрение разработанного решения в технологический стек потенциальных пользователей решения
    \item публикация исходного кода разработанного решения
\end{itemize}
\end{enumerate}

Этапы разработки подробно раскрыты ниже.

\section{подготовка аппаратной платформы решения}
бла бла

\section{разработка необходимого программного обеспечения}
бла бла
\section{отладка разработанного решения}
бла бла
\section{доработка решения с учётом требований потенциальных пользователей}
бла бла
\section{применение разработанного решения к реальной задаче}
бла бла
\section{публикация законченного разработанного решения как готового}
бла бла

\section{Обоснование принципов действия разработанных объектов}
\subsection{Архитектура разработанного решения}

\section{Характеристики разработанных объектов}
