\chapter*{Словарь основных понятий}
\thispagestyle{plain}
\paragraph*{Программно"=аппаратный комплекс} это совокупность программных и технических средств, спроектированных совместно работать над решением одной общей задачи или нескольких общих задач.

\paragraph*{Когнитивность} это свойство объекта, проявляющееся в способности к  

\paragraph*{Диалог} это процесс общения между двумя агентами на некотором языке.

\paragraph*{Робот} это устройство, способное автоматизированно выполнять некоторое множество задач.

\paragraph*{Когнитивный диалоговый робот}

\paragraph*{Текущее состояние} это характеристика, описывающая состояние системы, существующей во времени, в некотором временном промежутке.

\paragraph*{Реактивность} это способность к осуществлению некоторых действий --- реакций --- вследствие совершения некоторых событий, принадлежащего предопределённому множеству событий, порождающих некоторые реакции, не используя концепцию текущего состояния.
% https://www.cs.cmu.edu/afs/cs/usr/pstone/public/papers/97MAS-survey/node14.html
\paragraph*{Делиберативность} это способность к осуществлению некоторых действий --- реакций --- вследствие выполнения некоторых условий, используя концепцию текущего состояния.

\paragraph*{Реактивно"=делиберативный подход} это способность к осуществлению некоторых действий --- реакций --- вследствие выполнения некоторых условий как с использованием концепции текущего состояния, так и без использования концепции текущего состояния.

\paragraph*{Микроконтроллер}

\paragraph*{Одноплатный компьютер}

\paragraph*{Облако}

\paragraph*{Облачный сервис}

\paragraph*{Тонкий клиент}

\paragraph*{Реакция}

\paragraph*{Технологический стек}
