% \addcontentsline{toc}{prettypart}{СПИСОК ТЕРМИНОВ И СОКРАЩЕНИЙ}
\prettypart{СПИСОК ТЕРМИНОВ И СОКРАЩЕНИЙ}
\paragraph*{Программно"=аппаратный комплекс} это совокупность программных и технических средств, спроектированных совместно работать над решением одной общей задачи или нескольких общих задач.

\paragraph*{Когнитивность} это свойство агента, проявляющееся в способности к восприятию внешней информации и работе с ней.

\paragraph*{Диалог} это процесс общения между двумя агентами на некотором языке.

\paragraph*{Робот} это устройство, способное автоматизированно выполнять некоторое множество задач.

\paragraph*{Когнитивный диалоговый робот} это робот, наделённый некоторыми когнитивными функциями и способный вести диалог.

\paragraph*{Текущее состояние} это характеристика, описывающая состояние системы, существующей во времени, в некотором временном промежутке.

\paragraph*{Реактивность} это способность к осуществлению некоторых действий --- реакций --- вследствие совершения некоторых событий, принадлежащего предопределённому множеству событий, порождающих некоторые реакции, не используя концепцию текущего состояния.
% https://www.cs.cmu.edu/afs/cs/usr/pstone/public/papers/97MAS-survey/node14.html
\paragraph*{Делиберативность} это способность к осуществлению некоторых действий --- реакций --- вследствие выполнения некоторых условий, используя концепцию текущего состояния.

\paragraph*{Реактивно"=делиберативный подход} это способность к осуществлению некоторых действий --- реакций --- вследствие выполнения некоторых условий как с использованием концепции текущего состояния, так и без использования концепции текущего состояния.

\paragraph*{Тонкий клиент} это клиент в клиент"=серверной архитектуре, переносящий существенную часть работы на сервер.

\paragraph*{Технологический стек} это совокупность технологий, используемых в работе кем"=то.

\paragraph*{USB} это универсальная последовательная шина, интерфейс для подключения перефирийных устройств к вычислительной технике.

\paragraph*{Ethernet} это популярная технология проводных локальных сетей.

\paragraph*{UWP"=приложение} это приложение, созданное для того, чтобы работать в универсальной платформе Windows (Universal Windows Platform).

\paragraph*{Логи} это множества записей журнала событий, лог"=запись --- одна запись в журнале событий.

\paragraph*{Система контроля версий} это система, регистрирующая изменения в некотором множестве файлов, предоставляя возможность отслеживать историю изменений.

\paragraph*{Пины} это низкоуровневый интерфейс ввода-вывода прямого управления, присутствующий, например, в устройствах Arduino и Raspberry Pi.