\documentclass[a4paper,14pt]{extreport} %размер бумаги устанавливаем А4, шрифт 14пунктов
\usepackage[T2A]{fontenc}
\usepackage[utf8]{inputenc} 
\usepackage[english,russian]{babel}
\usepackage{amssymb,amsfonts,amsmath,cite,enumerate,float} % подключаем нужные пакеты расширений
\usepackage{graphicx} % хотим вставлять в диплом рисунки
\usepackage{longtable} % для таблиц длиннее, чем на страницу
\usepackage{misccorr} % Добавляем "точки" после названий разделов
\usepackage{geometry} % Меняем поля страницы
\usepackage{indentfirst}%отступ первого абзаца после названия главы
\usepackage{setspace}
\usepackage{fancyhdr} 
\usepackage{titlesec} % название главы по центру


\graphicspath{{images/}}



\makeatletter
\renewcommand{\@biblabel}[1]{#1.} % Заменяем библиографию с квадратных скобок на точку
\makeatother

\geometry{left=2.5cm}% левое поле
\geometry{right=2cm}% правое поле
\geometry{top=2cm}% верхнее поле
\geometry{bottom=2cm}% нижнее поле

\setcounter{secnumdepth}{4}% уровень вложенности счётчиков

\onehalfspacing

% \titleformat{\chapter}{\centering\hyphenpenalty=10000\normalfont\bfseries}{
% \thechapter. }{0pt}{\normalfont}
% \makeatother
% \setlength{\headheight}{17pt}

\titleformat{\chapter}
{\normalfont\bfseries\centering}{\thechapter}{1em}{\normalfont\bfseries}

\titleformat{\section}
{\normalfont\bfseries\centering}{\thesection}{1em}{\normalfont\bfseries}

\titleformat{\subsection}
{\normalfont\bfseries\centering}{\thesubsection}{1em}{\normalfont\bfseries}

\titleformat{\subsubsection}
{\normalfont\bfseries\centering}{\thesubsubsection}{1em}{\normalfont\bfseries}


 
% Настройка вертикальных и горизонтальных отступов
\titlespacing*{\chapter}{0pt}{-30pt}{8pt}
\titlespacing*{\section}{\parindent}{*4}{*4}
\titlespacing*{\subsection}{\parindent}{*4}{*4}



% нумерация справа вверху
\pagestyle{fancy} 
\lhead{} \chead{} \rhead{\normalsize\thepage}
\lfoot{} \cfoot{} \rfoot{}
\renewcommand{\headrulewidth}{0pt}
\renewcommand{\footrulewidth}{0pt}
\fancypagestyle{plain}{%
    \fancyhf{} % clear all header and footer fields
    \fancyhead[RE,RO]{\normalsize \thepage} % Even page, Odd page; Right, Left, Center
    \renewcommand{\headrulewidth}{0pt}
    \renewcommand{\footrulewidth}{0pt}
}

% \AtBeginDocument{%
  \addtocontents{toc}{\protect\thispagestyle{plain}}%
  \addtocontents{lof}{\protect\thispagestyle{plain}}%
% }



%полуторный интервал между строками
%\renewcommand{\baselinestretch}{1.5}% вроде как плохой метод, т.к. растягивает заголовки и сноски, чего лучше не делать


\renewcommand{\contentsname}{Содержание} 
\setcounter{page}{2}


\setlength{\parindent}{1cm}

\begin{document}


\title{Программно"=аппаратный комплекс для~создания диалоговых когнитивных роботов на~основе реактивно"=делиберативного подхода}
\author{oserikov}
\maketitle% это титульный лист
\chapter*{РЕФЕРАТ}
Выпускная квалификационная работа содержит: \pageref{LastPage}\ страниц, \totalfigures\ рисунков, 2 источника.

ИСКУССТВЕННЫЙ ИНТЕЛЛЕКТ, КОГНИТИВНЫЕ ТЕХНОЛОГИИ, ДИАЛОГОВЫЕ СИСТЕМЫ, РЕАКТИВНЫЙ ПОДХОД, ДЕЛИБЕРАТИВНЫЙ ПОДХОД, ОБЛАЧНЫЕ ТЕХНОЛОГИИ, MICROSOFT AZURE, ИНТЕРНЕТ ВЕЩЕЙ, NOSQL.

В данной работе поставлена и решена задача создания программно"=аппаратного комплекса для создания диалоговых когнитивных роботов на~основе реактивно"=делиберативного подхода.

В теоретической части работы приведены рассуждения, предшествующие постановке задачи, оценена актуальность задачи и исследованы предпосылки к постановке задачи.

В практической части работы описан разработанный подход к решению задачи с использованием реактивно"=делиберативного подхода, поддерживаемого облачными технологиями. Рассмотрены ключевые элементы как аппаратной части решения, так и программной части.
\pagebreak
\tableofcontents
\pagebreak

\chapter*{Словарь основных понятий}
\thispagestyle{plain}
\paragraph*{Программно"=аппаратный комплекс} это совокупность программных и технических средств, спроектированных совместно работать над решением одной общей задачи или нескольких общих задач.

\paragraph*{Когнитивность} это свойство объекта, проявляющееся в способности к  

\paragraph*{Диалог} это процесс общения между двумя агентами на некотором языке.

\paragraph*{Робот} это устройство, способное автоматизированно выполнять некоторое множество задач.

\paragraph*{Когнитивный диалоговый робот}

\paragraph*{Текущее состояние} это характеристика, описывающая состояние системы, существующей во времени, в некотором временном промежутке.

\paragraph*{Реактивность} это способность к осуществлению некоторых действий --- реакций --- вследствие совершения некоторых событий, принадлежащего предопределённому множеству событий, порождающих некоторые реакции, не используя концепцию текущего состояния.
% https://www.cs.cmu.edu/afs/cs/usr/pstone/public/papers/97MAS-survey/node14.html
\paragraph*{Делиберативность} это способность к осуществлению некоторых действий --- реакций --- вследствие выполнения некоторых условий, используя концепцию текущего состояния.

\paragraph*{Реактивно"=делиберативный подход} это способность к осуществлению некоторых действий --- реакций --- вследствие выполнения некоторых условий как с использованием концепции текущего состояния, так и без использования концепции текущего состояния.

\paragraph*{Микроконтроллер}

\paragraph*{Одноплатный компьютер}

\paragraph*{Облако}

\paragraph*{Облачный сервис}

\paragraph*{Тонкий клиент}

\chapter{Введение}

\section{актуальность работы}
\section{значимость работы}
\subsection{теоретическая значимость работы}
\subsection{практическая значимость работы}
\section{объект работы}
\section{предмет работы}
\section{цель работы}
\section{задачи работы}
\section{исследование, выполненное в работе}
\subsection{методы исследования}
\subsection{информационная база исследования} 



\part*{ОСНОВНАЯ ЧАСТЬ}
\chapter{Теоретическая часть работы}

Про когнитивные сервисы, микроконтроллеры, экспертные системы.

\section{История вопроса}
\section{Обзор литературы по вопросу}
\section{Представление различных точек зрения}
\section{Позиция автора исследования}
\section{Анализ и классификация привлечённого материала}
\section{Описание процесса теоретических исследований}
\subsection{Выбор компонент}

\chapter{Практическая часть работы}

Вышеописанные исследования привели к чётко поставленной задаче --- реализовать программно-аппаратный комплекс, который позволял бы максимально доступным образом создавать диалоговых когнитивных роботов. Аппаратная часть комплекса должна быть основана на микроконтроллерах Arduino и микрокомпьютерах Raspberry Pi. Программная часть должна реализовывать логику робота, используя реактивно"=делиберативный подход.

\section{Основные этапы в разработке решения задачи, сформулированной в выпускной квалификационной работе}
Основными этапами разработки явились:
\begin{enumerate}
\item подготовка аппаратной платформы решения 
\begin{itemize}
    \item исследование микроконтроллеров Arduino и микрокомпьютеров Raspberry Pi, представляющих основу аппаратной части, с точки зрения программиста
    \item разработка единой аппаратной платформы, объединяющей Arduino и Raspberry Pi в единый элемент, реализующий поставленную задачу
\end{itemize}

\item разработка необходимого программного обеспечения
\begin{itemize}
    \item выбор технологий для построения программного решения поставленной задачи
    \item выделение ключевых подзадач поставленной задачи в самостоятельные единицы, подлежащие разработке
    \item разработка выделенных на предыдущем этапе элементов
\end{itemize}

\item отладка разработанного решения
\begin{itemize}
    \item отладка разработанного решения
\end{itemize}

\item доработка решения с учётом требований потенциальных пользователей
\begin{itemize}
    \item анализ разработанного решения потенциальными пользователями
    \item доработка разработанного решения с учётом требований потенциальных пользователей
\end{itemize}

\item применение разработанного решения к реальной задаче  
\begin{itemize}
    \item реализация пробного проекта вместе с потенциальными пользователями решения
    \item получение обратной связи от потенциальных пользователей разработанного решения и сторонних компетентных специалистов
\end{itemize}

\item публикация законченного разработанного решения как готового
\begin{itemize}
    \item внедрение разработанного решения в технологический стек потенциальных пользователей решения
    \item публикация исходного кода разработанного решения
\end{itemize}
\end{enumerate}

Этапы разработки подробно раскрыты ниже.

\section{подготовка аппаратной платформы решения}
бла бла

\section{разработка необходимого программного обеспечения}
бла бла
\section{отладка разработанного решения}
бла бла
\section{доработка решения с учётом требований потенциальных пользователей}
бла бла
\section{применение разработанного решения к реальной задаче}
бла бла
\section{публикация законченного разработанного решения как готового}
бла бла

\section{Обоснование принципов действия разработанных объектов}
\subsection{Архитектура разработанного решения}

\section{Характеристики разработанных объектов}


\chapter{Заключение}
\section{Положения, выносимые на защиту}
\subsection{Общие результаты работы}
\subsection{Выводы, сделанные в процессе работы}
\subsection{Предположения, сделанные в процессе работы}
\section{Дальнейшее развитие}
\subsection[Перспективы применения результатов работы \\на практике]{Перспективы применения результатов работы на практике}
\subsection{Возможности дальнейшего исследования проблемы}



\begin{thebibliography}{9}
\bibitem{lang-krugosvet} 
\texttt{http://www.krugosvet.ru/enc/gumanitarnye\_nauki/lingvistika/YAZIK.html}
 
\bibitem{einstein} 
Albert Einstein. 
\textit{Zur Elektrodynamik bewegter K{\"o}rper}. (German) 
[\textit{On the electrodynamics of moving bodies}]. 
Annalen der Physik, 322(10):891–921, 1905.
 
\bibitem{knuthwebsite} 
Knuth: Computers and Typesetting,
\\\texttt{http://www-cs-faculty.stanford.edu/\~{}uno/abcde.html}
\end{thebibliography}



\end{document} 


